\def\probno{C}
\def\probtitle{등차수열의 합}

\section{\probno{}. \probtitle{}}

\begin{frame} % No title at first slide
    \sectiontitle{\probno{}}{\probtitle{}}
    \sectionmeta{
        \texttt{math}\\
        출제진 의도 -- \textbf{\color{acbronze}Easy}
    }
    \begin{itemize}
        \item 처음 푼 팀: \textbf{스콘빨리먹기대회우승팀}, 5분
        \item 처음 푼 팀(Open Contest): \textbf{sedev57}, 2분
        \item 출제자: 오주원
    \end{itemize}
\end{frame}

\begin{frame}{\probno{}. \probtitle{}}
    \begin{itemize}
        \item 두 등차수열의 합은 등차수열이어야 합니다.
        \item 따라서 $A$가 등차수열일 때만 $A_i=B_i+C_i$인 두 등차수열 $B, C$가 존재합니다.
        \item $B, C$를 만드는 것은 여러 방법이 있지만, $B_i = A_i, C_i = 0$으로 만드는 것이 가장 간단합니다.
    \end{itemize}
\end{frame}
