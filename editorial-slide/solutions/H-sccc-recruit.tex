\def\probno{H}
\def\probtitle{SCCC 신입 부원 모집하기}

\section{\probno{}. \probtitle{}}

\begin{frame} % No title at first slide
    \sectiontitle{\probno{}}{\probtitle{}}
    \sectionmeta{
        \texttt{dp\_bitfield}\\
        출제진 의도 -- \textbf{\color{acgold}Hard}
    }
    \begin{itemize}
        \item 처음 푼 팀: \textbf{스콘빨리먹기대회우승팀}, 140분
        \item 처음 푼 팀(Open Contest): \textbf{aeren}, 40분
        \item 출제자: 박찬솔
    \end{itemize}
\end{frame}

\begin{frame}{\probno{}. \probtitle{}}
    \begin{itemize}
        \item 세 가지 풀이가 존재합니다.

        \begin{enumerate}
            \item 동적 계획법을 이용한 $O(NKX^K)$ 풀이
            \item 최소 비용 최대 유량을 이용한 $O(N^3K + N^2K^2)$ 풀이
            \item 이분 그래프 최대 매칭을 이용한 $O(N^2KX)$ 풀이
        \end{enumerate}

        \item 차례대로 설명합니다.
    \end{itemize}
\end{frame}

\begin{frame}{\probno{}. \probtitle{} - DP}
    \begin{itemize}
        \item 점수가 높은 사람부터 차례대로 배정합니다.
        \item 각 스터디 그룹 $i$의 정원이 $C_i$만큼 남았을 때, 현재 인원의 상태를 $C = (C_1, C_2, \cdots, C_K)$라는 배열로 표현할 수 있습니다.
        \item 아래 DP를 채운 다음, 배정 방법이 존재하는 $D[N][\ast]$를 아무거나 선택해서 출력하면 됩니다.
        \begin{itemize}
            \item $D[i][C] = i$번째 사람까지 배정해서 인원의 상태가 $C$가 되게 하는 최적의 배정이 존재하는가? 존재한다면 그러한 배정 방법을 아무거나 하나 저장한다.
        \end{itemize}
        \item $i$의 범위는 $[1, N]$이고 $C$의 각 원소의 범위는 $[0, X]$ 범위이므로, $C$는 $X+1$진법을 이용해 $0$ 이상 $(X+1)^K$ 미만의 정수 값으로 표현할 수 있습니다.
        \item 따라서 상태의 개수는 총 $O(NX^K)$개입니다.
        \item 각 상태마다 $O(K)$개의 다음 상태가 있으므로 전체 시간 복잡도는 $O(NKX^K)$ 입니다.
        
    \end{itemize}
\end{frame}

\begin{frame}{\probno{}. \probtitle{} - 최소 비용 최대 유량}
    \begin{itemize}
        \item 문제에서 제시한 스터디 그룹 배정 알고리즘은 다음과 같은 성질을 갖고 있습니다.

        \begin{enumerate}
            \item 배정되는 사람의 수를 최대화함
            \item 배정된 사람을 내림차순으로 정렬했을 때 사전 순 최대인 결과를 만들어 냄
        \end{enumerate}
        \vspace{3mm}
        \item 점수가 $i$번째로 높은 사람이 매칭되었을 때의 가중치를 $-2^{N-i}$로 두면
        \item 사람을 최대한 많이 매칭시키면서 가중치를 최소화하는 문제가 됩니다.
        \item 따라서 이 문제는 최소 비용 최대 유량을 이용해 해결할 수 있습니다.
        \item 정점이 $O(N+K)$개, 간선이 $O(NK)$개, 최대 유량은 $N$이므로 전체 시간 복잡도는 $O(N \times NK \times (N+K)) = O(N^3K + N^2K^2)$입니다.
    \end{itemize}
\end{frame}

\begin{frame}{\probno{}. \probtitle{} - 이분 그래프 최대 매칭}
    \begin{itemize}
        \item 최소 비용 최대 유량 풀이에서 사용한 성질을 그대로 이용합니다.

        \begin{enumerate}
            \item 배정되는 사람의 수를 최대화함
            \item 배정된 사람을 내림차순으로 정렬했을 때 사전 순 최대인 결과를 만들어 냄
        \end{enumerate}
        \vspace{3mm}
        \item DFS를 이용해 $i$번째 정점을 매칭에 추가할 수 있는지 확인할 수 있습니다.
        \item 따라서 (배정된 사람 집합) + \{$i$번째 사람\}의 배정이 가능한지 $O(NKX)$에 확인할 수 있습니다.
        \item 이 과정을 $N$번 수행하므로 전체 시간 복잡도는 $O(N^2KX)$입니다.
    \end{itemize}
\end{frame}