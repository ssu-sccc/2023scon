\def\probno{B}
\def\probtitle{팀명 정하기}

\section{\probno{}. \probtitle{}}

\begin{frame} % No title at first slide
    \sectiontitle{\probno{}}{\probtitle{}}
    \sectionmeta{
        \texttt{math, implementation, sorting}\\
        출제진 의도 -- \textbf{\color{acbronze}Easy}
    }
    \begin{itemize}
        \item 처음 푼 팀: \textbf{반드시 가야지요}, 8분
        \item 처음 푼 팀(Open Contest): \textbf{asdf1705}, 4분
        \item 출제자: 나정휘
    \end{itemize}
\end{frame}

\begin{frame}{\probno{}. \probtitle{}}
    \begin{itemize}
        \item 첫 번째 팀명은 입학 연도를 오름차순으로 정렬하고 100으로 나눈 나머지를 출력하면 됩니다.
        \item 두 번째 팀명은 팀원을 문제 수 내림차순으로 정렬한 다음 성씨의 첫 글자를 출력하면 됩니다.
        \item 팀원이 항상 3명만 주어지기 때문에 조건문만 이용해도 문제를 해결할 수 있습니다.
    \end{itemize}
\end{frame}
