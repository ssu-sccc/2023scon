\def\probno{A}
\def\probtitle{정보섬의 대중교통}

\section{\probno{}. \probtitle{}}

\begin{frame} % No title at first slide
    \sectiontitle{\probno{}}{\probtitle{}}
    \sectionmeta{
        \texttt{math, implementation}\\
        출제진 의도 -- \textbf{\color{acbronze}Easy}
    }
    \begin{itemize}
        \item 처음 푼 팀: \textbf{반드시 가야지요}, 2분
        \item 처음 푼 팀(Open Contest): \textbf{asdf1705}, 1분
        \item 출제자: 박찬솔
    \end{itemize}
\end{frame}

\begin{frame}{\probno{}. \probtitle{}}
    \begin{itemize}
        \item $N \leq B$ 이므로 항상 지하철을 놓치지 않고 이용할 수 있습니다.
        \item 따라서 $N$의 값은 정답에 영향을 미치지 않습니다.
        \item $A < B$ 이면 \texttt{Bus}, $A > B$ 이면 \texttt{Subway}, $A = B$ 이면 \texttt{Anything}을 출력하면 됩니다.
    \end{itemize}
\end{frame}
