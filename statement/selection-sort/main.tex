\def\probtitle{선택 정렬의 이동 거리}
\def\probno{D}

\begin{problem}{\probno{}. \probtitle{}}

$1$부터 $N$까지의 정수가 한 번씩 등장하는 수열 $A$가 주어진다. 이 수열에서 선택 정렬 알고리즘을 수행할 때, 각 수의 이동 거리를 출력하라.

선택 정렬 알고리즘이 무엇인지 잘 모르는 친구들은 친절한 주원이가 준비한 아래 설명을 읽어보도록 하자. \\

\begin{itemize}
\item 길이가 $N$인 수열 $A = \left\{A_1, A_2, \cdots, A_N \right\}$을 오름차순으로 정렬하는 선택 정렬 알고리즘은 아래 동작을 $N-1$번 반복해서 수행한다.

\begin{enumerate}[noitemsep,topsep=0pt]
    \item 지금이 $i$번째 동작이라면, $A_i, A_{i+1}, \cdots, A_N$ 중 최솟값 $A_j$를 찾는다.
    \item $A_i$와 $A_j$의 위치를 교환한다. 이때 $A_i$와 $A_j$의 이동 거리가 각각 $(j-i)$만큼 증가한다. 
\end{enumerate}
\end{itemize}

예를 들어 $\left\{ 1, 3, 5, 2, 4 \right\}$와 같은 수열이 주어졌다고 하자. 처음에 모든 수의 이동 거리는 $0$으로 같다. 선택 정렬 알고리즘은 다음과 같은 과정을 거쳐 수행된다.

\begin{enumerate}[noitemsep]
    \item $A_1=1$과 $A_1=1$을 교환해서 $\left\{ 1, 3, 5, 2, 4 \right\}$가 된다. 이때 $1$의 이동 거리는 $0$만큼 증가한다.
    \item $A_2=3$과 $A_4=2$를 교환해서 $\left\{ 1, 2, 5, 3, 4 \right\}$가 된다. 이때 $2$와 $3$의 이동 거리는 $2$만큼 증가한다.
    \item $A_3=5$와 $A_4=3$을 교환해서 $\left\{ 1, 2, 3, 5, 4 \right\}$가 된다. 이때 $3$과 $5$의 이동 거리는 $1$만큼 증가한다.
    \item $A_4=5$와 $A_5=4$를 교환해서 $\left\{ 1, 2, 3, 4, 5 \right\}$가 된다. 이때 $4$와 $5$의 이동 거리는 $1$만큼 증가한다.
\end{enumerate}

따라서 $1$은 $0$만큼, $2$는 $2$만큼, $3$은 3만큼, $4$는 $1$만큼, $5$는 $2$만큼 이동한다.

\InputFile

첫째 줄에 수열의 길이 $N$이 주어진다.

둘째 줄에 수열의 원소 $A_1, A_2, \cdots, A_N$이 차례대로 공백으로 구분되어 주어진다.

\OutputFile

첫째 줄에 $N$개의 정수를 공백으로 구분하여 출력한다. $i$번째 정수는 $i$의 이동 거리를 의미한다.

\Constraints

\begin{itemize}[noitemsep]
    \item $1 \leq N \leq 5 \times 10^5$
    \item $A$에는 $1$부터 $N$까지의 정수가 정확히 한 번씩 등장한다.
\end{itemize}

\Example

\begin{example}
    \exmpfile{./example/01.in.txt}{./example/01.out.txt}%
    \exmpfile{./example/02.in.txt}{./example/02.out.txt}%
\end{example}

% \newpage

% \Notes


\end{problem}