\def\probtitle{정보섬의 대중교통}
\def\probno{A}

\begin{problem}{\probno{}. \probtitle{}}

숭실대학교 정보과학관은 숭실대입구역으로부터 멀리 떨어져 있는 대신, 바로 앞에 \textit{숭실대별관앞}이라는 명칭의 버스 정류소가 자리 잡고 있다.

학부 연구생 찬솔이는 야근을 마치고 대중교통을 이용해 집에 가려고 한다. 다행히 아슬아슬하게 막차가 끊기지 않은 상황인데, 구체적으로 $A$분 뒤에 숭실대별관앞 정류소에 집으로 가는 마지막 버스가 도착하고, $B$분 뒤에 숭실대입구역에 집으로 가는 마지막 열차가 도착한다.

찬솔이는 지금 버스 정류소에 서 있다. 그런데, 찬솔이는 지하철 역까지 걸어가서 지하철을 타는 것이 여기서 버스를 타는 것 보다 빠를 수도 있다는 사실을 알아차려 버렸다. 숭실대입구역의 지하철 승강장까지 걸어가는데는 $N$분이 걸린다. 버스와 지하철 중 더 먼저 탈 수 있는 것이 무엇인지 알려줘서 야근한 찬솔이의 피로를 회복시켜 주자.

단, 버스와 지하철이 도착한 뒤 탑승하는 데 걸리는 시간은 신경 쓰지 않고, 버스와 지하철 모두 도착한 직후에 승객을 태운 뒤 기다리지 않고 바로 떠난다. 또한 지하철 역에 도착하는 시간과 지하철 열차가 도착하는 시간이 정확히 같다면 지하철을 탈 수 있다.

\InputFile

첫째 줄에 $N, A, B$가 공백으로 구분되어 주어진다.

\OutputFile

버스에 더 먼저 탑승할 수 있으면 \t{Bus}, 지하철에 더 먼저 탑승할 수 있으면 \t{Subway}, 버스와 지하철에 탑승하게 되는 시간이 동일하면 \t{Anything}을 출력한다.

\Constraints

\begin{itemize}[noitemsep]
    \item $1 \leq A \leq 10^6$
    \item $1 \leq N \leq B \leq 10^6$
    \item 입력으로 주어지는 수는 모두 정수이다.
\end{itemize}

\Example

\begin{example}
    \exmpfile{./example/01.in.txt}{./example/01.out.txt}%
    \exmpfile{./example/02.in.txt}{./example/02.out.txt}%
    \exmpfile{./example/03.in.txt}{./example/03.out.txt}%
\end{example}

첫 번째 예시에서 버스는 5분 후에 탑승할 수 있고, 지하철은 10분 동안 역에 걸어간 다음 5분을 더 기다려야 하므로 15분 후에 탑승할 수 있다.

% \newpage

% \Notes


\end{problem}