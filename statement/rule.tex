{
    \indent
    \Large
    
    대회 중 유의 사항
}

\begin{itemize}[noitemsep]
    \item 2023 숭실대학교 프로그래밍 대회(2023 SCON) 유의 사항입니다.
    
    \item 대회 정보
    
    \begin{itemize}[noitemsep]
        \item 이 대회는 숭실대학교 IT대학이 주최, 컴퓨터학부 문제해결 소모임 SCCC가 주관합니다.
        \item 이 대회는 현대모비스와 스타트링크의 후원을 받아 진행됩니다.
    \end{itemize}
    
    \item 사진 촬영 안내
    
    \begin{itemize}[noitemsep]
        \item 대회 당일 현장 스태프가 대회 현장을 촬영하여 사진으로 기록 및 온라인 게시할 예정입니다.
        \item 사진은 추후 SCCC의 홍보와 SCCC가 주관하는 대회 홍보에 사용될 수 있습니다.
    \end{itemize}

    \item 대회 진행 관련

    \begin{itemize}[noitemsep]
        \item 대회는 3시간 동안 10문제로 진행됩니다.
        \item 6문제 이상 해결한 팀은 스코어보드 갱신이 중단됩니다.
        \item 대회 종료 1시간 전부터 모든 팀의 스코어보드 갱신이 중단됩니다.
        \item 대회 도중에는 대회 스태프와 팀원을 제외한 타인과 대화할 수 없습니다.
        \item 대회 사이트 및 언어 공식 레퍼런스 사이트를 제외한 모든 인터넷의 사용은 금지됩니다.

        \begin{itemize}[noitemsep]
            \item \texttt{BOJ Help \ : https://help.acmicpc.net/language/info}
            \item \texttt{C/C++ \ \ \ \ : https://en.cppreference.com/w/}
            \item \texttt{Java \ \ \ \ \ : https://docs.oracle.com/en/java/javase/15/docs/api/}
            \item \texttt{Python \ \ \ : https://docs.python.org/3/}
        \end{itemize}

        \item 화장실 이용 시 스태프와 동행해야 하며, 한 번에 한 명씩 이용 가능합니다.
        \item 대회 도중 휴대전화 사용이 불가능합니다. 불가피한 경우 스태프의 감독하에 사용 가능합니다.
        \item 문제와 관련된 질문은 대회 페이지의 `질문' 탭을 이용해야 합니다. 현장 스태프는 문제에 대한 질문을 받지 않습니다.
        \item 대회 공지는 대회 페이지의 `공지' 탭을 이용해 전달합니다. 주기적으로 확인해 주시길 바랍니다.
    \end{itemize}

    \item 문제 관련

    \begin{itemize}[noitemsep]
        \item 문제는 운영진들이 생각하는 난이도순으로 정렬되어 있습니다.
        \item 모든 문제는 C++과 Java로 해결할 수 있음이 보장됩니다.
        \item 제출한 프로그램은 문제에 명시된 제한 시간 내에 정답을 출력하고 정상적으로 종료되어야 합니다. 이는 return code가 0이어야 함을 의미하여, 이외의 exit code는 런타임 에러가 발생합니다.
        \item 제출한 프로그램은 문제에 명시된 제한 메모리보다 많은 메모리를 사용할 수 없습니다.
        \item 언어별 추가 시간과 추가 메모리가 주어지지 않습니다.
        % \item 모든 문제의 메모리 제한은 1024MiB로 동일합니다.
        \item 제출한 프로그램은 표준 입력(standard input)을 통해 입력받아서 표준 출력(standard output)을 통해 정답을 출력해야 합니다.
        \item 표준 입출력을 제외한 파일 입출력, 네트워킹, 멀티 스레딩 등의 시스템 콜은 사용할 수 없습니다.
    \end{itemize}

    \item 팀의 등수는 다음과 같은 방법을 이용해 계산합니다.

    \begin{itemize}[noitemsep]
        \item 문제의 패널티 = (대회가 시작한 시점으로부터 처음으로 \textbf{\textcolor{acgreen}{맞았습니다!!}}를 받기까지 걸린 분 단위 시간) + (제출 횟수 - 1) * 20분
        \item 팀의 패널티 = \textbf{\textcolor{acgreen}{맞았습니다!!}}를 받은 모든 문제의 패널티의 합
        \item 팀의 등수 = (더 많은 문제를 푼 팀의 수) + (푼 문제의 개수가 동일하면서 패널티가 더 작은 팀의 수) + 1
        \item 컴파일 에러는 패널티에 포함되지 않습니다.
    \end{itemize}
\end{itemize}

{
    \indent
    \Large    
    간식 안내
}

\begin{itemize}[noitemsep]
    \item 제공되는 간식에 알레르기 유발 물질이 포함되어 있으니 주의해서 섭취하시길 바랍니다.
    \item 영양성분표 및 원재료명 관련 문의는 대회 페이지의 질문 기능을 통해 질문하면 답변드리겠습니다.
    \item 오예스
    \begin{itemize}[noitemsep]
        \item 우유, 밀, 계란, 대두 함유
        \item 땅콩을 사용한 제품과 같은 제조시설에서 제조
    \end{itemize}
    \item 트윅스
    \begin{itemize}[noitemsep]
        \item 밀, 우유, 대두 함유
        \item 헤이즐넛 혹은 아몬드를 사용한 제품과 같은 제조시설에서 제조
    \end{itemize}
    \item 쿠크다스 케이크
    \begin{itemize}[noitemsep]
        \item 밀, 우유, 대두, 계란, 돼지고기 함유
    \end{itemize}
    \item 비쵸비 비스켓
    \begin{itemize}[noitemsep]
        \item 밀, 우유, 대두, 쇠고기 함유
        \item 달걀, 땅콩, 호두, 복숭아, 토마토, 돼지고기, 닭고기, 오징어, 새우, 게, 조개류(굴, 홍합) 혼입 가능
    \end{itemize}
\end{itemize}