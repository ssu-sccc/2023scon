\def\probtitle{등차수열의 합}
\def\probno{C}

\begin{problem}{\probno{}. \probtitle{}}

수학과 전공과목인 조합론을 수강하는 정휘는 등차수열의 합 공식에 대해 배우고 있다. 2023 SCON 대회 개최가 일주일 남았지만, 아직 문제를 절반도 만들지 못해 발등에 불이 떨어진 정휘는 화장실에 가는 척을 하면서 정보과학관에 달려와 등차수열에 관한 문제를 만들었다.

길이가 $N$인 수열 $A$가 주어졌을 때, $1 \le i \le N$에 대해 $A_i = B_i + C_i$를 만족하고 길이가 $N$인 두 등차수열 $B, C$를 구하라.

등차수열의 정의는 다음과 같다.
\begin{itemize}[noitemsep,topsep=0pt]
    \item 어떤 수열 $A = \left\{ A_1, A_2, \cdots, A_N \right\}$이 등차수열이라는 것은, $2 \leq i \leq N$인 모든 $i$에 대해 $A_i - A_{i-1}$이 모두 동일한 수열을 말한다. 정의에 따라 길이가 $2$ 이하인 수열은 항상 등차수열이다.
    % \item 등차수열이란 $2 \le i \le N$인 모든 $i$에 대해 $A_{i}-A_{i-1}$이 일정한 수열을 말한다. 길이가 $2$ 이하인 수열은 등차수열이다.
\end{itemize}

\InputFile

첫째 줄에 수열 $A$의 길이 $N$이 주어진다.

둘째 줄에 수열 $A$의 원소 $A_1, A_2, \cdots, A_N$이 순서대로 공백으로 구분되어 주어진다.

\OutputFile

만약 모든 $1 \leq i \leq N$에 대해 $A_i = B_i + C_i$인 길이가 $N$인 두 등차수열 $B, C$가 존재하지 않으면 첫째 줄에 \t{NO}를 출력한다.

그렇지 않으면 첫째 줄에 \t{YES}를 출력한다. 이후 둘째 줄에 $B$의 원소를, 셋째 줄에 $C$의 원소를 차례대로 공백으로 구분해서 출력한다. 가능한 수열 $B, C$가 여럿인 경우, 아무거나 하나만 출력한다. 수열 $B, C$가 존재할 경우, 문제의 제한을 만족하는 출력이 존재한다는 것을 증명할 수 있다.

\Constraints

\begin{itemize}[noitemsep]
    \item $1 \leq N \leq 10^5$
    \item $-10^9 \leq A_i \leq 10^9$ $(1 \le i \le N)$ 
    \item $-10^{18} \leq B_i, C_i \leq 10^{18}$ $(1 \le i \le N)$ 
    \item $A_i$, $B_i$, $C_i$는 모두 정수 $(1 \le i \le N)$
\end{itemize}

\Example

\begin{example}
    \exmpfile{./example/01.in.txt}{./example/01.out.txt}%
    \exmpfile{./example/02.in.txt}{./example/02.out.txt}%
    \exmpfile{./example/03.in.txt}{./example/03.out.txt}%
\end{example}

% \newpage

% \Notes


\end{problem}