\def\probtitle{팀명 정하기}
\def\probno{B}

\begin{problem}{\probno{}. \probtitle{}}

\begin{quote}
    현대 모비스는 모빌리티 SW 해커톤, 알고리즘 경진대회, 채용 연계형 SW 아카데미 등 다양한 SW 인재 발굴 프로그램을 진행하고 있다. 지난 2월에 개최된 모빌리티 SW 해커톤은 국내 14개 대학의 소프트웨어 동아리 20개 팀, 70여 명이 참여해 모빌리티 소프트웨어 개발 실력을 겨뤘다.
\end{quote}

숭실대학교 컴퓨터학부 문제해결 소모임 SCCC 부원들은 매년 모빌리티 SW 해커톤, SCON, ICPC와 같은 팀 대회에서 사용할 팀명을 정하기 위해 많은 고민을 한다. 졸업을 한 학기 남겨둔 성서는 더 이상 부원들이 팀명으로 고통을 받지 않도록 가이드라인을 만들었다.

성서의 가이드라인에 따르면 팀 이름을 짓는 방법은 두 가지가 있다.

\begin{enumerate}[noitemsep]
    \item 세 참가자의 입학 연도를 100으로 나눈 나머지를 오름차순으로 정렬해서 이어 붙인 문자열
    \item 세 참가자 중 성씨를 영문으로 표기했을 때의 첫 글자를 백준 온라인 저지에서 해결한 문제가 많은 사람부터 차례대로 나열한 문자열
\end{enumerate}

예를 들어 600문제를 해결한 18학번 안(AHN)씨, 2000문제를 해결한 19학번 이(LEE)씨, 6000문제를 해결한 20학번 오(OH)씨로 구성된 팀을 생각해 보자. 첫 번째 방법으로 팀명을 만들면 181920이 되고, 두 번째 방법으로 팀명을 만들면 OLA가 된다.

2000문제를 해결한 19학번 이(LEE)씨, 9000문제를 21학번 나(NAH)씨, 1000문제를 해결한 22학번 박(PARK)씨로 구성된 팀은 첫 번째 방법으로 팀명을 만들면 192122가 되고, 두 번째 방법으로 팀명을 만들면 NLP가 된다.

세 팀원의 백준 온라인 저지에서 해결한 문제의 개수, 입학 연도, 그리고 성씨가 주어지면 첫 번째 방법과 두 번째 방법으로 만들어지는 팀명을 차례대로 출력하는 프로그램을 작성하라.

\InputFile

첫째 줄에 첫 번째 팀원이 백준 온라인 저지에서 해결한 문제의 개수 $P_1$, 입학 연도 $Y_1$, 성씨 $S_1$이 공백으로 구분되어 주어진다.

둘째 줄과 셋째 줄에는 두 번째 팀원의 정보 $P_2, Y_2, S_2$와 세 번째 팀원의 정보 $P_3, Y_3, S_3$이 첫째 줄과 같은 형식으로 주어진다.

\OutputFile

첫째 줄에 첫 번째 방법으로 만든 팀명을 출력한다.

둘째 줄에 두 번째 방법으로 만든 팀명을 출력한다.

\newpage

\Constraints
\begin{itemize}[noitemsep]
    \item $1 \leq P_i \leq 20\,000$ $(1 \le i \le 3)$
    \item $P_i$는 모두 서로 다르다. $(1 \le i \le 3)$
    \item $2010 \leq Y_i \leq 2099$ $(1 \le i \le 3)$
    \item $Y_i$는 모두 서로 다르다. $(1 \le i \le 3)$
    \item 성씨는 알파벳 대문자로만 구성되어 있으며, 최대 5글자이다.
    \item 모든 성씨는 서로 다르다.
    \item 입력으로 주어지는 수는 모두 정수이다.
\end{itemize}

\Example

\begin{example}
    \exmpfile{./example/01.in.txt}{./example/01.out.txt}%
    \exmpfile{./example/02.in.txt}{./example/02.out.txt}%
\end{example}

% \newpage

\Notes

181920은 2020 ICPC Asia Seoul Regional Contest 동상 수상팀, NLP는 2022 ICPC Asia Seoul Regional Contest 은상 수상팀이다.


\end{problem}